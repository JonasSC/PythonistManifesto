%%%%%%%%%%%%%%%%%%%%%%%%%%%%%%%%%%%%%%%%%%%%%%%%%%%%%%%%%%%%%%%%%%%%%%%%%%%%%%%%
% Copyright 2015 Jonas Schulte-Coerne                                          %
%                                                                              %
% Licensed under the Creative Commons Attribution-ShareAlike license           %
% version 4.0 or later (https://creativecommons.org/licenses/by-sa/4.0/)       %
%%%%%%%%%%%%%%%%%%%%%%%%%%%%%%%%%%%%%%%%%%%%%%%%%%%%%%%%%%%%%%%%%%%%%%%%%%%%%%%%

\chapter{Tipps for the advanced Python-beginner}
	This manual is aimed at people, who already know the syntax to define a class or a function in Python.
	With this it is possible to solve most programming problems, so that certain functionalities of Python, that might facilitate this task, are sometimes overlooked.
	This chapter introduces some of these functionalities.

	\section{Exception handling}
		Never trust a program, that doesn't crash, when it is about to enter an invalid state!

		Exceptions can be raised, when the program is about to enter a state, which the programmer cannot or does not want to implement.
		Normally this happens, when invalid or incompatible data is given, like an unparseable string or a division by zero.
		Another reason to raise an exception is, when the program and its dependencies are not properly set up.
		This could mean missing libraries or missing configuration files.
		Sometimes, exceptions are raised because of algorithmic problems, for example when the maximum recursion depth is exceeded.
		And then there are hardware related exceptions, for when a device is not responding or when the main memory is full.

		Exceptions are not only for stopping the program, before it destroys anything.
		They are also for facilitating the debug process by giving a hopefully informative error message and by showing where the error occured.
		Nothing is harder to debug than an error, that happened at a completely different place, than where the exception has been raised.
		So it is generally a good idea to raise an exception instead of further interpreting the data, as soon as something looks vaguely like an error.

		It is also possible to raise warnings, that do not stop the program.
		This can be used for example to inform about deprecated features of the program.

		Here is a short list of good practices in raising and handling exceptions:
		\begin{itemize}
			\item Do not guess!\\
			      If choosing the proper action for a given piece of data is not a deterministic and reliable process, raise an error about ambiguous data.
			      A potentially wrong interpretation of the data will lead to a misbehaving program, where the errors are spotted way further down the processing chain, than the place, where the errors have been made.\\
			      Also a complicated maze of conditional expressions, that just interprets some data, is tedious to program, hard to understand and unlikely to be maintained properly.
			\item Raise meaningful exceptions!\\
			      In order to facilitate proper exception handling and maybe debugging aswell, do not just raise an {\normalfont \ttfamily Exception}.
			      Please bother to at least raise a {\normalfont \ttfamily RuntimeError}, if it is not clear, which exception shall be raised.
			      But if possible, raise a more specific exception like an {\normalfont \ttfamily IOError}, a {\normalfont \ttfamily ValueError}, an {\normalfont \ttfamily IndexError}, a {\normalfont \ttfamily ZeroDivisionError} or a {\normalfont \ttfamily NotImplementedError}.\\
			      There are many more predefined exceptions listed in the Python documentation.
			      If they are not specific enough, it is possible to define custom errors by implementing a class that inherits from {\normalfont \ttfamily BaseException}.
			\item Catch exceptions with precision!\\
			      Don't just program a general {\normalfont \ttfamily try ... except} block to mute some error message.
			      Maybe , there are unsuspected exceptions, that are raised inside the {\normalfont \ttfamily try} block, which are then masked, because the {\normalfont \ttfamily try} block just catches any exception.
			      This again leads to errors, that occur later than when they happen.\\
			      So at least specify, which type of exception shall be caught, by naming the class in the {\normalfont \ttfamily except} block.
			      It is possible to catch different types of exceptions, by appending multiple {\normalfont \ttfamily except} blocks to one {\normalfont \ttfamily try} block.\\
			      One can also get the object of the raised exception by typing ''{\normalfont \ttfamily except EXCEPTION\_CLASS as e}''.
			      This allows a further disambiguation of the caught exception, for example by parsing its error message, and reraising the exception again, when it shall not be caught.
		\end{itemize}


	\section{Neat modules in Python's standard library}
		The Python standard library offers some very useful modules, that are often overlooked by programmers, who have just started with Python.
		Some of these modules are not only targeted for very specific applications, but they are just helpful in general.
		These modules shall be introduced in this section.

		\subsection{logging}
			\label{logging}
			The logging module offers an alternative to using {\normalfont \ttfamily print} for status messages.
			The advantage of the logging module is, that it offers some functionality for specifying the handling of the status messages without modifying the code that generates the status messages.

			When writing code, that generates a status message, the programmer has to decide, with which log-level the message shall be generated.
			A reasonable set of log-levels is already implemented in the logging module, but it is also possible to specify custom ones.
			The already implemented log-levels are {\normalfont \ttfamily DEBUG}, {\normalfont \ttfamily INFO}, {\normalfont \ttfamily WARNING}, {\normalfont \ttfamily ERROR} and {\normalfont \ttfamily CRITICAL}.
			They can be used with convenient functions from the logging module.
			\begin{minted}[gobble=4, tabsize=4]{python}
				import logging
				logging.debug("This is a debug message")
				logging.info("Consider yourself informed")
				logging.warning("I warned you not to use the print statement")
				logging.error("Not obeying the rules in manual is an error")
				logging.critical("This is a message of critical importance")
			\end{minted}

			With the logging module it is possible to specify which messages shall be displayed and which shall be skipped.
			The following line of code sets up, that all debug- and info-messages are dropped, while everything with the priority of a warning and more is displayed.
			\begin{minted}[gobble=4, tabsize=4]{python}
				logging.getLogger().setLevel(logging.WARNING)
			\end{minted}

			Furthermore it is easily possible to redirect the logger's output to a file or a widget in a graphical user interface.
			\begin{minted}[gobble=4, tabsize=4]{python}
				handler = logging.FileHandler("file.log") # writes to a file
				handler = logging.StreamHandler()         # writes to a stream, like the console
				                                          # output or a text field in a GUI
				handler = logging.NullHandler()           # discards everything
				logging.getLogger().addHandler(handler)
			\end{minted}
			Those handlers can be customized with formatters, to specify, which information shall be logged in addition to the log message.
			This information can be for example the log-level, the file or the line number in which the logging has been invoked.

			It is also possible to create a separate logger for each module.
			This way, the logging can be customized in a very fine grained way, by specifying a log level for each module individually.
			All loggers inherit their log-level from the root-logger, so a common use for this is to set the root-Logger to a very high log-level, while switching on the debug messages in one specific module.
			The following code shows how to create and use a module-specific logger.
			\begin{minted}[gobble=4, tabsize=4]{python}
				import logging
				# create Logger object
				logger = logging.getLogger(__name__)
				# send a logging message
				logger.info("Hello World")	# "logger" not "logging"
			\end{minted}
			The logger object can now be set up independently from the root logger, by specifying a log-level or adding a Handler.

			Please remember that logging is just for sending status messages.
			It is not an alternative to raising an exception or even giving an exception a meaningful error message.

		\subsection{collections}
			The collections module defines some interesting classes for data containment.
			One of them is the {\normalfont \ttfamily OrderedDict}, which behaves like a normal dictionary, but when iterating over its keys, the iteration will be performed in the order in which the items have been added to the {\normalfont \ttfamily OrderedDict}.
			This is a very convenient mapping type for data, whose order is important.

			Other classes that are provided by the collections module are general abstractions, that can be used with the {\normalfont \ttfamily isinstance} function to check, if an object has a certain functionality.
			The following code lines test, if the object ''{\normalfont \ttfamily a}'' is iterable or callable:
			\begin{minted}[gobble=4, tabsize=4]{python}
				import collections
				i = isinstance(a, collections.Iterable) # True, if a is a list,
				                                        # a tuple, a set, a dict,
				                                        # or anything else, that
				                                        # can be iterated in a
				                                        # for loop.
				c = isinstance(a, collections.Callable) # True, if a can be called
				                                        # like a function or a
				                                        # method.
			\end{minted}
			This behavior of {\normalfont \ttfamily isinstance} shows again, that Python uses duck typing (see section~\ref{DuckTyping} for more about this).
			The compared object's type does not have to have inherited from an iterable class, so {\normalfont \ttfamily isinstance} recognizes it as a subclass of {\normalfont \ttfamily collections.Iterable}.
			It just has to have the functionality of an iterable object.
			The behavior is accordingly with {\normalfont \ttfamily collections.Callable}.

		\subsection{functools}
			The functools module provides functions and decorators to enhance the capabilities of callable objects like functions or methods.

			One interesting function, provided by this module, is {\normalfont \ttfamily partial}.
			It can be used to create a new function by defining certain arguments from a base function.
			This technique is sometimes referred to as ''Currying'' or ''Schönfinkeln''\footnote{After its inventors Gottlob Frege, Moses Schönfinkel and Haskell Brooks Curry. So sadly ''Currying'' does not come from ''spicing up a function''.}.
			An example for using this function could be to create an {\normalfont \ttfamily increase}-function from an {\normalfont \ttfamily add}-function:
			\begin{minted}[gobble=4, tabsize=4]{python}
				import functools

				def add(a, b):	# define an add-function, that adds two values
					return a + b

				increase = functools.partial(add, b=1)	# this function adds 1 to a

				y = increase(4)	# y = 4+1
			\end{minted}

			In Python3 there exists a special function {\normalfont \ttfamily functools.partialmethod} for creating methods with a simplified signature:
			This function is not available in the Python2 version of functools.
			\begin{minted}[gobble=4, tabsize=4]{python}
				class Light(object):
					def __init__(self):
						self.__switched_on = False

					def Switch(self, switch_on):
						self.__switched_on = switch_on

					SwitchOn = functools.partialmethod(Switch, switch_on=True)
					SwitchOff = functools.partialmethod(Switch, switch_on=False)
			\end{minted}

			Another useful feature of the functools module is the decorator\footnote{read more about decorators in subsection~\ref{Decorators}.} {\normalfont \ttfamily lru\_cache}.
			This decorator is also only available in Python3.
			This decorator can be used to cache return values of a function, so it needs not run again, if it has already run for the given set of arguments.
			This can save computing time on functions that take a long time to run, but are only called with a small variety of arguments.
			\begin{minted}[gobble=4, tabsize=4]{python}
				import functools
				import time

				# "maxsize" is the maximum number of cached values. Calling the
				# function with a new combination of parameters will result in the
				# deletion of the result for least recently used combination.
				# "typed" specifies, if the arguments' types shall be considered.
				# If False, intense_computing(a=1.0, b=2.0) will be mapped to the
				# same result as intense_computing(a=1, b=2).
				@functools.lru_cache(maxsize=2, typed=False)
				def intense_computing(a, b):
					time.sleep(3)	# simulate some really hard work!
					return a + b

				v = intense_computing(9.0, 6.0)  # this will do some hard work
				w = intense_computing(2.3, 4.1)  # this will do some hard work, too
				x = intense_computing(9, 6)      # this will just use the cached value
				y = intense_computing(3.5, 7.1)  # this will do some hard work again
				z = intense_computing(2.3, 4.1)  # this will do some hard work, because
				                                 # the cached value has been deleted
				                                 # after the last function call
			\end{minted}

		\subsection{inspect}
			The inspect module provides functionalities to get information about objects like functions or classes.
			For example, it allows retrieving the arguments of a function (including their default values) or it can deliver the inheritance hierarchy of a class.
			The features of this module are so extensive, that they cannot be discussed in the briefness of this manual.
			But they are thoroughly explained in the official documentation.

			Object introspection can be cumputationally expensive, so it is advised not to use the inspect functionalities in performance critical code sequences.
			Most programming tasks can do without it, anyway.\\
			But in certain situations, the inspect module can be extremely helpful.
			For example, when serializing an object in order to save it to a file.
			Or for producing valuable debug information, when raising an error.


	\section{Special Python expressions}
		Python provides some expressions, for which the need is not immediately obvious, since their functionality can be emulated with basic elements of the Python language.
		Nevertheless, they are extremely helpful for writing cleaner code, that might run faster and is easier to maintain and understand.

		\subsection{with}
			Certain objects need to be destroyed manually after using them.
			For example a filestream has to be closed after extracing all necessary information from the file.\\
			The manual destruction can be tedious, especially when it has to be ensured with {\normalfont \ttfamily try} and {\normalfont \ttfamily finally} blocks.\\
			So Python allows to specify the initialization and destruction methods {\normalfont \ttfamily \_\_enter\_\_} and {\normalfont \ttfamily \_\_exit\_\_}, which can be invoked with the {\normalfont \ttfamily with} statement.
			The {\normalfont \ttfamily with} statement creates a block, in which an object can be used.
			It takes care of calling the {\normalfont \ttfamily \_\_enter\_\_} method when entering the block and of calling the {\normalfont \ttfamily \_\_exit\_\_} method when leaving the block.
			The {\normalfont \ttfamily \_\_exit\_\_} method is always called, even when an exception is raised.

			The following example shows how to use the {\normalfont \ttfamily with} statement to access a file:
			\begin{minted}[gobble=4, tabsize=4]{python}
				with open("path_to_file", "r") as f:    # f is the variable name, by
					for l in f.readlines():             # which the file object can
						print(l.strip())                # be accessed
				# no need to call f.close(). "with" will do that.
			\end{minted}

			To open multiple files, simply separate them with commas in the with statement.
			In order to avoid overly long lines, a backslash can be used to state that the following line belongs to the current command.
			\begin{minted}[gobble=4, tabsize=4]{python}
				with open("path_to_file1", "r") as f1,\
				     open("path_to_file2", "w") as f1,\
				     open("path_to_file3", "wb") as f3:
					pass	# do something with the files
			\end{minted}

		\subsection{yield}
			{\normalfont \ttfamily yield} is an alternative to the {\normalfont \ttfamily return} statement for functions or methods that return sequences of items.
			First creating the whole sequence and then returning it can consume a lot of memory, so {\normalfont \ttfamily yield} allows to return the sequence element wise.
			This way, it is possible to iterate over the return value of a function, while only having the current element in memory.
			The future elements are not loaded yet and the past elements can be (or have been) discarded.\\
			The return value of such a function is called a ''generator''.

			The {\normalfont \ttfamily yield} statement is typically located inside a loop.
			The following example shows a function that returns a list and the equivalent implementation, employing the {\normalfont \ttfamily yield} statement.
			\begin{minted}[gobble=4, tabsize=4]{python}
				def with_return(maximum):
					result = []
					for i in range(maximum + 1):
						result.append(i)
					return result

				def with_yield(maximum):
					for i in range(maximum + 1):
						yield i
			\end{minted}

			Since the elements of the returned sequence of a function utilizing the {\normalfont \ttfamily yield} statement are discarded, when advancing to the next element, random access to the sequence's elements is not possible.
			For accessing the elements randomly, they have to be loaded into the memory by casting the result value into a randomly accessible data type like a tuple or a list.
			\begin{minted}[gobble=4, tabsize=4]{python}
				a = with_return(maximum)[3]         # this will work
				b = with_yield(maximum)[3]          # this will crash
				c = tuple(with_yield(maximum))[3]   # this will work
			\end{minted}

			Many functions of python return a generator or generator like objects.
			For example the {\normalfont \ttfamily range} function or the {\normalfont \ttfamily readlines} method of a filestream.

		\subsection{lambda}
			The {\normalfont \ttfamily lambda} statement allows defining functions like a variable.
			This functionality is not absolutely necessary, since Python allows to define functions even inside other functions (closures) and the handling of function pointers is easy.
			However for simple functions, that just calculate a value in one line of code, a brief {\normalfont \ttfamily lambda} statement clutters the code less than the definition of a closure.

			Lambda functions are created by assigning a {\normalfont \ttfamily lambda} statement, which is followed by the function's arguments and its return value, to a variable.
			The return value is just stated by the formula that calculates it; the {\normalfont \ttfamily return} statement is not issued.
			The following example shows how to define a lambda function:
			\begin{minted}[gobble=4, tabsize=4]{python}
				# the usual way:
				def add(a, b):
					return a + b

				# with lambda:
				add = lambda a, b: a + b
			\end{minted}


	\section{Advanced Python tricks}
		This section introduces some cool aspects of Python, that increase its usability tremendously, although they might not be necessary to accomplish everyday programming tasks.

		\subsection{Features of the interpreter}
			The Python interpreter makes use of two important environment varibles~\footnote{Environment variables can be set with the commands {\normalfont \ttfamily set} or {\normalfont \ttfamily setx} on Windows and {\normalfont \ttfamily export} on Unix and its derivates.}.
			\begin{itemize}
				\item {\normalfont \ttfamily PYTHONPATH} can be used to specify an additional path, in which Python will search for modules, when trying to import something.
				      The environment variable is usually empty and is used for making libraries available, that are not installed system wide, when executing a specific program that needs them.
				      This is useful for testing programs or libraries that are still in development.
				      Usually such programs are started through a Bash/Batch-script, that sets the {\normalfont \ttfamily PYTHONPATH} accordingly before executing the program.\\
				      {\normalfont \ttfamily PYTHONPATH} can contain multiple paths.
				      On Windows they are separated by semicolons, on Linux with colons.
				\item {\normalfont \ttfamily PYTHONSTARTUP} can be used to specify the path to a script, that is executed, when the Python interpreter is started in interactive commandline mode.
				      This can be used to import modules that are commonly used like {\normalfont \ttfamily math} or {\normalfont \ttfamily os}.\\
				      The following lines are a startup script, that enables autocompletion for the interactive Python interpreter:
				      \begin{minted}[gobble=11, tabsize=4]{python}
				      	import readline
				      	import rlcompleter
				      	readline.parse_and_bind("tab: complete")
				      \end{minted}
				      It might be necessary to install the readline package before using this script.
				      When this script has run, hitting the tab-key no longer enters a tab, but it requests the autocompletion.
				      Hitting tab once extends the currently entered start of a command by the letters of a command that can be determined unambigously.
				      Hitting tab twice shows a list of all possible extensions to the currently entered start of a command.
			\end{itemize}

			A program can be started in interactive mode with the commandline parameter {\normalfont \ttfamily -i}.
			\begin{center}
				{\normalfont \ttfamily python -i my\_program.py}
			\end{center}
			With this, program is run normally and after it has finished, the Python interpreter switches to the interactive command line mode.
			This can be very useful for debugging, as all global variables, that are instantiated by the program, can then be investigated interactively.\\
			Of course global variables should have been avoided, so one will rarely start a whole program in interactive mode.
			But if the software parts are written in a nicely modular way, one can write a script that instantiates the part that shall be debugged, and then run it in order to test that software part interactively.

		\subsection{Decorators}
			\label{Decorators}
			Decorators are a special feature of Python to enhance functions, methods and classes with extra functionality.
			They are functions themselves, that operate with the decorated function as a parameter.
			To apply a decorator to a function, the decorator is noted with an {\normalfont \ttfamily @} as a prefix.

			The following example shows a decorator that ensures that the return value of a function is never zero.
			\begin{minted}[gobble=4, tabsize=4]{python}
				def non_zero(function):
					def new_function(*args, **kwargs):
						result = function(*args, **kwargs)
						if result == 0.0:
							return 1e-12
						else:
							return result
					return new_function

				@non_zero
				def subtract(a, b):
					return a - b
			\end{minted}
			When interpreting this code, Python will replace the {\normalfont \ttfamily subtract}-function with the return value of {\normalfont \ttfamily non\_zero(subtract)}.
			This will be the closure {\normalfont \ttfamily new\_function}, whose variable {\normalfont \ttfamily function} is set to {\normalfont \ttfamily subtract}.
			A function call of {\normalfont \ttfamily subtract} will be replaced like the following:
			\begin{equation}
				subtract(a, b) \Rightarrow non\_zero(subtract)(a, b) \Rightarrow new\_function(a, b)
			\end{equation}

			The decorator separates the securing of a nonzero result from the actual calculation.
			And it can of course be applied to other functions aswell.
			This way, decorators can be a very elegant vehicle to comply with the ''Do only one thing, and do it well'' mantra.\\
			But decorators are often hard to understand, so be careful, not to program in accordance to the Monty-Python slogan ''surprise your friends, amuse your enemies''.

			The function, with which the original function is replaced needs not be created directly by the decorator function.
			When using a closure, for creating the replacing function, decorators with parameters become possible.
			The following example shows a decorator that limits a functions return value to a specifiable maximum.
			\begin{minted}[gobble=4, tabsize=4]{python}
				def limit(maximum):
					def generator(function):
						def new_function(*args, **kwargs):	# did you watch Inception?
							result = function(*args, **kwargs)
							if result > maximum:
								return maximum
							else:
								return result
						return new_function
					return generator

				@limit(5)
				def add(a, b):
					return a + b
			\end{minted}

			A decorator only needs to be a callable object, so decorators with parameters are usually much easier to understand, if they are implemented in form of a class:
			\begin{minted}[gobble=4, tabsize=4]{python}
				class Limit(object):
					def __init__(self, maximum):
						self.__maximum = maximum
						self.__function = None

					def __ReplacementFunction(self, *args, **kwargs):
						result = self.__function(*args, **kwargs)
						if result > maximum:
							return maximum
						else:
							return result

					def __call__(self, function):   # defining this makes
						self.__function = function  # the object callable
						return self.__ReplacementFunction
			\end{minted}

			Of course, decorators can be stacked:
			\begin{minted}[gobble=4, tabsize=4]{python}
				@Limit(5)
				@non_zero
				def add(a, b):
					return a + b
			\end{minted}
			In this example, the input function for calling the {\normalfont \ttfamily Limit}-object is the replaced function from the {\normalfont \ttfamily non\_zero}-decorator.

			When decorating a method, the decorators will be called with the unbound methods, since they are executed when creating the class and not an object of that class.
			Often, when decorating a method, the decorator shall modify a bound method, so it needs information about the object, when it is created.
			For this, it is possible to define a class with a {\normalfont \ttfamily \_\_get\_\_}-method, which is executed, when the method of an object is accessed.
			These classes are called ''descriptors''.
			This method gets the instance and the class as parameters and shall return the replacement function.
			Be careful when storing a reference to the instance, as this might result in cyclic references (see section~\ref{GarbageCollection} for more on this).
			The following example shows the usage of a decorator for a method:
			\begin{minted}[gobble=4, tabsize=4]{python}
				class Descriptor(object):
					def __init__(self, unbound_method):
						pass

					def __get__(instance, owner):
						def replacement_function(*args, **kwargs): # changes nothing
							return self.__unbound_method(instance, *args, **kwargs)
						return replacement_function

				def decorator(method):	# this can be used with @decorator
					return Descriptor(method)
			\end{minted}
			The replacements that are going on in this example are shown below.\\
			When interpreting the class:
			\begin{equation}
				\begin{aligned}
					            & MyClass.Method\\
					\Rightarrow & decorator(MyClass.Method)\\
					\Rightarrow & Descriptor(MyClass.Method)
				\end{aligned}
			\end{equation}
			When calling a method of an object:
			\begin{equation}
				\begin{aligned}
					            & myObject.Method(arg) \\
					\Rightarrow & Descriptor(MyClass.Method).\_\_get\_\_(myObject, MyClass)(arg) \\
					\Rightarrow & replacement\_function(arg)
				\end{aligned}
			\end{equation}

		\subsection{Profiling}
			Python has inbuilt capabilities for profiling a Python program.
			Profiling means, running the program and logging the time, that the interpreter spends on any given command.
			These logs reveal performance issues and parts, where performance optimizations would be most beneficial.

			The profiling modules can be imported in the software's source code for a very fine grained setup of which commands shall be logged.
			For more about this, please read the manual page about this subject~\cite{Profiling}, as documenting these extensive capabilities would be beyond the scope of this manual.

			There is also a way of profiling a complete program, without modifying it.
			This can be done by loading the {\normalfont \ttfamily cProfile} module at the startup of the interpreter, before executing the program.
			\begin{center}
				{\normalfont \ttfamily python -m cProfile -o profile\_file my\_program.py}
			\end{center}
			In this example, the path where the profiling log is stored is given by {\normalfont \ttfamily -o profile\_file}.

			Profiling can be used to compare the performance of different implementations.
			For the sake of fairness, the programs, that are to be profiled for this purpose, must only differ in that implementation.
			This also includes, that the programs have to be deterministic, which means:
			\begin{itemize}
				\item no user interaction
				\item no random numbers, that are not seeded to be equal in both programs
				\item no interaction with hardware, whose performance might vary a lot (e.g. network communication).
			\end{itemize}

			The profiling logs can then be evaluated in a Python script with the help of the {\normalfont \ttfamily pstats} module.
			\begin{minted}[gobble=4, tabsize=4]{python}
				import pstats

				# load the profiling log file
				p_loaded = pstats.Stats("profile_file")
				# sort by the time spent in the given function
				p_sorted = p_loaded.sort_stats("time")
				# print information about the 20 functions that used the most time
				p_sorted.print_stats(20)
			\end{minted}
			In this example, the functions are sorted by the time that has actually been spend in that function.
			When a function called another function, only the time of calling the other function is added to the first function's time.
			If the runtime of the other function shall also be added to the time of the calling function, the profile must be sorted by the cumulative time.
			\begin{minted}[gobble=4, tabsize=4]{python}
				...
				p_sorted = p_loaded.sort_stats("cumulative")
				...
			\end{minted}


	\section{Differences between Python2 and Python3}
		Programming in Python2 and Python3 is very similar.
		The differences are mainly details and the underlying C implementation.
		This is the reason, why many libraries for Python2 have not been ported to Python3, yet.\\
		It is however possible to write Python code that runs with both Python2 and Python3.
		This section shall give an overview about the main differences and how to achieve compatibility with both versions.

		\subsection{print}
			Probably the most obvious difference between the two versions is, that {\normalfont \ttfamily print} is a function in Python3.
			Of couse typing the additional parentheses is annoying, but this is also a much cleaner programming interface, which can be advantagous in many situations.
			For example a pointer to the {\normalfont \ttfamily print} function can be passed as a handler, that specifies what to do with some data.
			\begin{minted}[gobble=4, tabsize=4]{python}
				def do_something(function, data):
					return function(data)

				do_something(function=print, data=(1, 2, 3))
			\end{minted}

			In order to use the function version of {\normalfont \ttfamily print} in Python2, one has to import it from (the) {\normalfont \ttfamily \_\_future\_\_}.
			\begin{minted}[gobble=4, tabsize=4]{python}
				from __future__ import print_function
				print("Hello World")
			\end{minted}
			Imports of {\normalfont \ttfamily \_\_future\_\_} features, that have been backported to Python2, are ignored in Python3, so that code which uses them, runs with both versions.

		\subsection{Imports}
			\label{Imports}
			Python3 tries to distiguish more between modules, that have been installed in some system directory and modules that are part of the software's source code.
			This is similar to the {\normalfont \ttfamily \#include} statement in C/C++, where {\normalfont \ttfamily \#include~''header.h''} specifies, that the header file shall be included from the local source code, while {\normalfont \ttfamily \#include~\textlangle header.h\textrangle} includes a system wide installed header.
			In this context, ''system wide'' means everything, that can be found through {\normalfont \ttfamily sys.path}, so even self written and locally installed libraries are considered to be installed ''system wide'' here.
			Since the working path, in which the Python interpreter has been started (independent of later calls of {\normalfont \ttfamily os.chdir}), can also be found through {\normalfont \ttfamily sys.path}, the modules in this path are also handled as if they were installed system wide.

			In Python2 all modules are being directly imported by importing their name, while in Python3, local modules are imported through the modules ''.'' and ''..''.
			System wide installed libraries are still imported by their name in Python3.\\
			Maybe an example can illustrate these confusing sentences.
			Assuming, one wants to import the function {\normalfont \ttfamily helperfunction} from the module {\normalfont \ttfamily helper}, which is defined in the file ''helper.py'' in the current working directory, the following lines would be used to import this function.
			\begin{minted}[gobble=4, tabsize=4]{python}
				# import a system wide library (works the same way with both versions)
				import os

				# Python2
				# import a module into its own namespace
				import helper
				# import a function into the local namespace
				from helper import helperfunction

				#Python3
				# import a module into its own namespace
				from . import helper
				# import a function into the local namespace
				from .helper import helperfunction
			\end{minted}

			The Python3 way of importing modules has some advantages.
			\begin{itemize}
				\item Python can now distinguish between local and system wide modules with the same name.
				\item It is possible to import modules that are in the parent directory of the current module (with {\normalfont \ttfamily from .. import module})
				\item It also works with Python2.7, so it is advised to use it for portable code.
			\end{itemize}

		\subsection{String / Unicode}
			In Python2, strings are usually ascii strings, while prepending a {\normalfont \ttfamily u} before the quotes defines, that the string shall be encoded as unicode (e.g. {\normalfont \ttfamily u''Hello~World''}).
			Python3 on the other hand always uses unicode.

			Due to its fixation on unicode, Python3 can do without some functions, that are available in Python2.
			These are among others the {\normalfont \ttfamily unicode} function, that is used to cast objects into unicode strings, and the {\normalfont \ttfamily BaseString} class, that is used to check, if a variable is a (unicode) string, with {\normalfont \ttfamily isinstance}.
			These have to be avoided, when writing code, that shall run with both versions.\\
			Also note, that the {\normalfont \ttfamily str} function behaves differently, as its Python2 variant converts its parameter to ascii, while in Python3 it is converted to unicode.
			This can be utilized for portable code, where an object shall be casted to the most common string format in the respective version of Python.

			Furthermore, Python3 is rather picky about its input.
			For example, when reading a text file, one has to specify the file encoding, when opening the file, if it is not unicode:
			\begin{minted}[gobble=4, tabsize=4]{python}
				with open("file.txt", "r", encoding="ISO-8859-15") as f:
					# do something with the file
			\end{minted}
			In Python2, the {\normalfont \ttfamily open} function does not accept an {\normalfont \ttfamily encoding} parameter, so it is not possible to open a non-unicode file with the same line of code in both versions of Python.
			Since it is hard write the {\normalfont \ttfamily with} line in a conditional way, that depends on the Python version, without duplicating the code inside the {\normalfont \ttfamily with} block, it might be best, to create the parameters for the {\normalfont \ttfamily open} function inside an {\normalfont \ttfamily if} block.
			\begin{minted}[gobble=4, tabsize=4]{python}
				import sys

				kwargs = {}
				if sys.version_info.major >= 3:
					kwargs["encoding"] = "ISO-8859-15"
				with open("file.txt", "r", **kwargs) as f:
					for line in f:
						if sys.version_info.major <= 2:
							line = line.decode("ISO-8859-15")
						# do something with the lines of the file
			\end{minted}

			If the encoding is not specified for a non-unicode file, Python3 crashes, while Python2 only reads peculiar special characters from non-ascii files.
			Actually Python3's behavior is a good thing, because it crashes, where the error happens.
			Python2 masks the error by going on, until the string is interpreted somewhere.

		\subsection{Integer divisions}
			This difference between Python2 and Python3 deserves special attention, as it is unlikely to lead to a crash, but it can easily lead to unintended calculation results.\\
			In Python3, the normal division with ''/'' is always a floating point division, while in Python2, dividing two integers performs an integer division, which discards the remainder.
			The following code gives an example for this:
			\begin{minted}[gobble=4, tabsize=4]{python}
				a = 10.0 / 4.0 # 2.5 in both versions (as expected)
				a = 10 / 4     # 2 in Python2, 2.5 in Python3
				a = 10 / 2     # 5 in Python2, 5.0 in Python3 (a float although both
				               #                               operands are integers
				               #                               and the remainder is 0)
			\end{minted}

			In both versions of Python the ''//'' operator is used for divisions, that discards the remainder of the division.
			If both operands are integers, using this operator also returns an integer.
			So in order to write code that is portable between the two versions, use ''//'' to perform an integer division.
			\begin{minted}[gobble=4, tabsize=4]{python}
				a = 10.0 // 4.0  # 2.0 in both versions
				a = 10 // 4      # 2 in both versions
			\end{minted}


	\section{Coding style}
		This is the point, where this manual becomes inconsistent.
		Although it proudly claims on its title page, that it aims to abolish a totalitarian regime, it promotes the usage of a programming language that is pretty totalitarian about its syntax.\\
		While truly libertarian languages like Perl execute anything from latin up to a demonic mixture of special characters and burnt out brain cells, Python has adopted the idea, that there should be only one solution for each problem.
		Because of this, it imposes many restrictions how the code has to be written (e.g. its strict indentation rules).\\
		The reason to choose such a strict syntax specification for Python becomes clear, when thinking about the main benefit of a totalitarian system:
		Even stupid people understand, what's going on.
		So reading and understanding a piece of Python source code, that someone else has written, is most of the times reasonably simple.
		The art is to achieve this simplicity, without oppressing programmer in his way to think, talk, act and, of course, design a program.
		And Python practices this art rather admirably.

		Even Python did not dare to impose a fully unified coding style on the programmer, which means that certain code constructions can be written in different ways.
		Some best practices for programming in Python are compiled in PEP~8~\cite{PEP8}.\\
		The reasons for most of the recommendations in PEP~8 are thoroughly explained, so it is best to follow them.
		Other recommendations are just a matter of taste.
		Since many other projects are programmed in compliance with PEP~8, it is good to follow these recommendations aswell, just for the sake of a recognizable programming style.

		Some points of PEP~8 are an issue of discussion though:
		\begin{itemize}
			\item Feel free to exceed a line length of 79 characters, if it improves the readability of the code.
			      Since many programmers no longer use command line editors on small screens with a small resolution, the 79 character limit seems a bit strict and old fashioned.
			      It is still a good guideline to split a function call or a string to multiple lines, when it exceeds this threshold, but in some cases (like heavily indented blocks), a relaxed adherence to the 79 character rule can significantly improve the readability and even the quality of the code, as Raymond Hettinger pointed out~\cite{HettingerPEP8}.
			\item The indentation with four spaces is messy and, with inappropriate text editors, even painful.
			      Nevertheless, if the written code shall be published, it is better to change to a decent text editor, than to switch to an indentation with tabs.
			      Python is extremely fussy about its indentation characters and since most Python source code is indented with four spaces, it is best to obey that convention.
			      Otherwise, developers, who are new to the project, are likely to introduce errors, by assuming, that the code was indented with four spaces.\\
			      The general opinion in discussions seems to be, that although many people prefer tab indentation over space indentation, consistency is valued much higher than personal preference on that topic~\cite{TabsVsSpaces}.
			      Jamie Zawinski even gives some technical reasons for avoiding tabs in source code documents~\cite{Zawinski}.
		\end{itemize}

